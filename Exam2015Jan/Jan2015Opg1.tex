Betragt dette regulære udtryk over alfabetet $\{e, f, d, x\}$:
\begin{align*}
    e(fd*)*x
\end{align*}
Ved antagelse, at
\begin{align*}
    e & \text{ svarer til } & enter\:bar\\
    f & \text{ svarer til } & fetch\:beer\\
    d & \text{ svarer til } & drink\:beer\\
    x & \text{ svarer til } & exit\:bar
\end{align*}
så beskriver det regulære udtryk strenge, der symboliserer scenarier for en gæst i fredagsbaren.
\subsubsection{Giv nogle eksempler på strenge der genkendes af det regulære udtryk samt en uformel beskrivelse}
Nogle eksempler på strenge der genkendes af det regulære udtryk er.
\begin{itemize}
    \item ex
    \item efx
    \item efdddddddddx
    \item efffffffx
    \item efdfdfdx
\end{itemize}
Sproget beskriver en mængde handlinger man kan tage i fredagsbaren, som har disse krav.
\begin{itemize}
    \item Man skal gå ind og ud
    \item Det er ikke påkrævet at drikke hvis man har hentet en øl
\end{itemize}
\subsubsection{Lav udtrykket om til en NFA}
Vi kan starte med at dele udtrykket op i mindre dele og repræsentere dem via en graf visning.\\
\begin{figure}[!ht]
    \centering
    \begin{tikzpicture}[shorten >=1pt,node distance=2cm,on grid,auto] 
       \node[state, initial] (q_1)   {$1$}; 
        \node[state,accepting] (q_2) [ right =of q_1] {$2$};
              
        \path[->] 
        (q_1) edge  node {$e$} (q_2)
        ;
    \end{tikzpicture}
    \caption{$e$}\label{fig:2015:examfignfae}
\end{figure}
\newpage
\begin{figure}[!ht]
    \centering
    \begin{tikzpicture}[shorten >=1pt,node distance=2cm,on grid,auto] 
       \node[state, initial,accepting] (q_1)   {$1$}; 
        \node[state,accepting] (q_2) [right  =of q_1] {$2$};
              
        \path[->] 
        (q_1) edge [bend left] node{$f$} (q_2)
        (q_2) edge [loop] node{$d$} (q_2)
        (q_2) edge [loop] node{$d$} (q_2)
        (q_2) edge [bend left] node{$\epsilon$} (q_1)
        ;

    \end{tikzpicture}
    \caption{$(fd*)*$}\label{fig:2015:examfignfafd}
\end{figure}
\begin{figure}[!ht]
    \centering
    \begin{tikzpicture}[shorten >=1pt,node distance=2cm,on grid,auto] 
       \node[state, initial] (q_1)   {$1$}; 
        \node[state,accepting] (q_2) [ right =of q_1] {$2$};
              
        \path[->] 
        (q_1) edge  node {$x$} (q_2)
        ;
    \end{tikzpicture}
    \caption{$s$}\label{fig:2015:examfignfax}
\end{figure}
Hvis vi sætter figur~\ref{fig:2015:examfignfae}~\ref{fig:2015:examfignfafd}~\ref{fig:2016:examfignfax} sammen så får vi følgende graf.
\begin{figure}[!ht]
    \centering
    \begin{tikzpicture}[shorten >=1pt,node distance=2cm,on grid,auto] 
        \node[state, initial] (q_1)   {$1$}; 
        \node[state] (q_2) [right=of q_1] {$2$};
        \node[state] (q_3) [right=of q_2] {$3$};
        \node[state] (q_4) [right=of q_3] {$4$};
        \node[state,accepting] (q_5) [ right =of q_4] {$5$};
              
        \path[->] 
        (q_1) edge node{$e$}(q_2)
        (q_3) edge [bend left] node{$\epsilon$}(q_2)
        (q_2) edge [bend left] node{$f$}(q_3)
        (q_3) edge [loop] node{$d$}(q_3)
        (q_3) edge node{$\epsilon$}(q_4)
        (q_2) edge [bend right, in=270, out=270] node{$\epsilon$}(q_4)
        (q_4) edge node{$x$}(q_5)
        ;

    \end{tikzpicture}
    \caption{The labeled NFA, representing $e(fd*)*x$}\label{fig:2015:examfignfa}
\end{figure}
\newpage
\subsubsection{Lav udtrykket om til en DFA}
For at konstruerer en DFA fra NFA'en skal vi inddele automaten i subsæt og se hvilke sæt referer til andre sæt.
\begin{figure}[!ht]
    \centering
    \begin{tabular}{c|ccccc}
            &$e$&$f$&$d$&$x$&NFA state\\\hline
        $S_1$&$\{2,4\}^{S_2}$&$\{\}$&$\{\}$&$\{\}$&\{1\}\\
        $S_2$&$\{\}$&$\{2,3,4\}^{S_3}$&$\{\}$&$\{5\}^{S_4}$&\{2,4\}\\
        $S_3$&$\{\}$&$\{\}^{S_3}$&$\{\}^{S_3}$&$\{5\}^{S_4}$&\{2,3,4\}\\
        $S_4$&$\{\}$&$\{\}$&$\{\}$&$\{\}$&\{5\}
    \end{tabular}
\end{figure}
Nu kan vi konstruerer DFA'en.
\begin{figure}[!ht]
    \centering
    \begin{tikzpicture}[shorten >=1pt,node distance=2cm,on grid,auto] 
        \node[state, initial] (q_1)   {$S_1$}; 
        \node[state] (q_2) [right=of q_1] {$S_2$};
        \node[state] (q_3) [ right =of q_2] {$S_3$};
        \node[state,accepting] (q_4) [ right =of q_3] {$S_4$};
              
        \path[->] 
            (q_1) edge node{$e$}(q_2)
            (q_2) edge  node{$f$}(q_3)
            (q_2) edge [bend right] node[below]{$x$}(q_4)
            (q_3) edge node{$x$}(q_4)
            (q_3) edge [loop] node[above]{$f,d$}(q_3)
        ;
    \end{tikzpicture}
    \caption{The labeled DFA, representing $k(l|v|h|)+s$}\label{fig:2016:examfigdfa}
\end{figure}
\subsubsection{Beskriv mængden af strenge ved et regulært udtryk}
\begin{align*}
    f(s|d)*
\end{align*}