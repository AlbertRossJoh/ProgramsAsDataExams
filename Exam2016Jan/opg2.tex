\subsubsection{Udvid typen expr}
\begin{lstlisting}[style=fsharp]
type expr = 
    ...
    | Ref of expr
    | Deref of expr
    | UpdRef of expr * expr
\end{lstlisting}
\subsubsection{Udvid typen value}
\begin{lstlisting}[style=fsharp]
type value = 
    | Int of int
    | Closure of string * string * expr * value env
    | RefVal of value ref
\end{lstlisting}
\newpage
\subsubsection{Udvid eval i higherfun}
Jeg har lavet følgende udvidelse til eval
\begin{lstlisting}[style=fsharp]
let rec eval (e : expr) (env : value env) : value =
    match e with
    ...
    | Ref e -> 
      eval e env |> ref |>  RefVal
    | Deref e -> 
      match eval e env with
      | RefVal v -> v.Value
      | _ -> failwith "You can only dereference reference types"
    | UpdRef(e1, e2) ->
      match eval e1 env with
      | RefVal v -> 
        let other = eval e2 env
        v.Value <- other
        other
      | _ -> failwith "You you tried to update a non reference type"
\end{lstlisting}
beskrivelsen af koden følger:
\begin{itemize}
    \item Ref - Vi evaluerer først udtrykket så vi får en værdi som vi kan putte ind i vores ref celle herefter putter vi den ind i en ref celle så vi ikke længere passerer selve værdien rundt men ene reference til den og til sidst pakker vi den i vore ''RefVal'' type.
    \item Deref - Da vi kun kan hente værdien fra en reference værdi skal vi tjekke om den rent faktisk har den rigtife type, hvis den har det returnere vi værdien, ellers fejler vi.
    \item Updref - Samme som ''Deref'' da vi teknisk set skal ''dereference'' den så vi kan få værdien, forskellen er at vi herefter opdaterer værdien inde i ref cellen, samt returnerer den nye værdi.
\end{itemize}
\begin{figure}[!ht]
    \begin{lstlisting}[style=fsharp]
let exam2_1 = 
  Let("x", Ref(CstI 2),
    Let("y", Var "x",
      Let("z", UpdRef(Var "y", CstI 42),
          Deref(Var "x")
      )
    )
  )
    \end{lstlisting}
    \caption{Eksempel 1}\label{fig:2016:udtryk1}
\end{figure}
\begin{figure}[!ht]
    \begin{lstlisting}[style=fsharp]
let exam2_2 = 
  Let("x", Ref(CstB true),
    Deref(Var "x")
  )
    \end{lstlisting}
    \caption{Eksempel 2}\label{fig:2016:udtryk2}
\end{figure}
\begin{figure}[!ht]
    \begin{lstlisting}[style=fsharp]
let exam2_3 = 
  Let("x", Ref(CstI 10),
    Let("y", Ref(CstI 32),
      Letfun("loop", "i",
        If(
          Prim("=", Deref(Var "x"), CstI 0),
          Deref(Var "y"),
          Let("z", 
            UpdRef(Var "y", 
              Prim("+",
                Deref(Var "y"), 
                CstI 1)
            ),
            Call(Var "loop", UpdRef(Var "x", Prim("-",Deref(Var "x"), CstI 1)))
          )
        ),
        Call(Var "loop", Deref(Var "x"))
      )
    )
  )
    \end{lstlisting}
    \caption{Eksempel 3}\label{fig:2016:udtryk3}
\end{figure}
\begin{figure}[!ht]
    \begin{lstlisting}[style=fsharp]
let exam2_4 = 
  Let("x", Ref(CstI 1),
    Let("y", Ref(CstI 1),
      UpdRef(Var "x", Prim("+", Deref(Var "x"), Deref(Var "y")))
    )
  )
    \end{lstlisting}
    \caption{Eksempel 4}\label{fig:2016:udtryk4}
\end{figure}
\subsubsection{Udvid lexer og parser}
\begin{figure}[!ht]
    \centering
\begin{lstlisting}[style=fsharp]
{
   let keyword s =
       match s with
       ...
       | "ref"  -> REF
       ...
}
   
   rule Token = parse
     ...
     | ":="            { UPD }
     | '!'             { DEREF }
     ...
\end{lstlisting}
    \caption{Funlex opdateringer}
    \label{fig:2016:fslex}
\end{figure}
\begin{figure}[!ht]
    \centering
\begin{lstlisting}[style=fsharp]
%token REF UPD DEREF
%left REF              // lowest precedence  
%left ELSE
%left EQ NE UPD
%left GT LT GE LE 
%left PLUS MINUS
%left TIMES DIV MOD 
%nonassoc NOT DEREF          // highest precedence

Expr:
  ...
  | DEREF Expr     { Deref($2)      }
  | REF Expr       { Ref($2)        }
  | Expr UPD Expr  { UpdRef($1, $3) }
;
\end{lstlisting}
    \caption{Funpar opdateringer}
    \label{fig:2016:parser}
\end{figure}
I lexeren ``FunLex'' bliver der tilføjet 3 nye tokens, hhv. \Verb|REF|, \Verb|UPD| samt \Verb|DEREF|.\\
I parseren ``FunPar'' bliver de definerede tokens tilføjet samt vurderet præcedens mæssigt. Her har jeg valgt at give \Verb|REF| den laveste præcedens da vi skal kunne definere en værdi med et hvert slags udtryk. Jeg vurderer at \Verb|DEREF| skal have højeste præcedens da man typisk binder den direkte til en variabel, og til sidst har jeg valgt at \Verb|UPD| skal have samme præcedens som \Verb|EQ|.
\subsubsection{Omskriv eksemplerne}
\begin{figure}[!ht]
    \begin{lstlisting}[style=fsharp]
let x = ref 2 in
  let y = x in
    let z = y := 42 in
      !x
    end
  end
end
    \end{lstlisting}
    \caption{Implementation af figur~\ref{fig:2016:udtryk1}}\label{fig:2016:imp1}
\end{figure}
\begin{figure}[!ht]
\begin{lstlisting}[style=fsharp]
let x = ref 2 in 
    (x := 3) + !x 
end
\end{lstlisting}
    \caption{Implementation af figur~\ref{fig:2016:udtryk2}}\label{fig:2016:imp2}
\end{figure}
\begin{figure}[!ht]
\begin{lstlisting}[style=fsharp]
let x = ref 10 in 
  let y = ref 32 in
    let loop i =  
        if !x = 0 then !y
        else 
          let z = y := !y+1 in
            loop (x:=!x-1)
          end
      in loop (!x)
    end
  end
end
\end{lstlisting}
    \caption{Implementation af figur~\ref{fig:2016:udtryk3}}\label{fig:2016:imp2}
\end{figure}
\begin{figure}[!ht]
\begin{lstlisting}[style=fsharp]
let x = ref 1 in 
  let y = ref 1 in
    x := !x+!y
  end
end
\end{lstlisting}
    \caption{Implementation af figur~\ref{fig:2016:udtryk4}}\label{fig:2016:imp2}
\end{figure}
\newpage
\begin{landscape}
    
\subsubsection{Lav typeinferens for udtryk}
\begin{figure}[!ht]
    \centering
\begin{mathpar}
        \inferrule*[left=$t_6$]{ 
            \inferrule*[left=$ref$]{ 
                \rho\vdash 2 : \vb{int}
            }{
                \rho\vdash\vb{ref 2} : \vb{int ref}
            }\\
            \inferrule*[right=$t_4$]{ 
                \inferrule*[left=$\left(\vb{:=}\right)$]{ 
                    \inferrule*[left=$t_3$]{
                        \rho\left(x\right)=\vb{int ref}
                    }{
                        \rho\vdash x :\vb{int ref}
                    }
                    \inferrule*[right=$t_1$]{ }{
                        \rho\vdash3:\vb{int}
                    }
                }{
                    \rho\vdash\vb{x := 3}:\vb{int}
                }
                \inferrule*[right=$\left(!\right)$]{ 
                    \inferrule*[right=$t_3$]{
                        \rho\left(x\right)=\vb{int ref}
                    }{
                        \rho\vdash x :\vb{int ref}
                    }
                }{
                    \rho\vdash\vb{!x}:\vb{int}
                }
            }{
                \rho\left[x\mapsto\vb{int ref}\right]\vdash\vb{(x := 3) + !x}:\vb{int}
            }
        }{
            \rho\vdash\vb{let x = ref 2 in (x := 3) + !x end}:\vb{int}
        }
\end{mathpar}   
    \caption{Typeinferens af udtrykket \Verb|let x = ref 2 in (x := 3) + !x end|}
    \label{fig:2016:typeinference}
\end{figure}
\end{landscape}
