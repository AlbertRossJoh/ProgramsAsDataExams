\subsubsection{}
Givet følgende kode
\begin{figure}[!ht]\label{fig:exVarEnv}
\begin{verbatim}
let exVarEnv : varEnv = 
  ([
    ("a", (Locvar 6, TypA(TypI, Some 2)));
    ("pn", (Locvar 3, TypA(TypP TypI, Some 1)));
    ("p", (Locvar 1, TypP TypI));
    ("n", (Locvar 0, TypI));
    ("g", (Glovar 0, TypI))
  ], 7)
\end{verbatim} 
  \caption{}
\end{figure}\\
Kan vi se at værdien 7 angiver stak dybden. 
\begin{itemize}
  \item a - har et lokalt offset på $6$ med typen array af ints som er på to elementer, da der også skal være plads til en header bliver dette til $3$ elementer der bliver allokeret til.
  \item pn - har et lokalt offset på $3$ med typen array af pointers til ins på længden 1, igen på grund af header skal der allokeres til $2$ elementer.
  \item p - har et lokalt offset på $1$ med typen pointer til int og dertil skal kun allokeret ét element.
  \item n - har et lokalt offset på $0$ med typen int, og skal kun have allokeret ét element.
  \item g - har en absolut adresse på $0$ med typen int.
\end{itemize}
Dette vil altså sige at den næste variabel der skal allokeres til skal starte på position $7$.