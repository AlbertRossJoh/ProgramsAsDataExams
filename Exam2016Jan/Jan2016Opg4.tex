\subsubsection{Udvid micro-c med interval check}
\begin{figure}[!ht]
\begin{FSharp}
rule Token = parse
  ...
  (*@\hl{| '.'             \{ DOT \}}@*)
  ...
  | _               { failwith "Lexer error: illegal symbol" }
\end{FSharp}
\caption{CLex.fsl}\label{fig:2016:opg4:clex}
\end{figure}
\begin{figure}[!ht]
\begin{FSharp}
...
%token CHAR ELSE IF INT NULL PRINT PRINTLN RETURN VOID WHILE DOT
...
%right ASSIGN             /* lowest precedence */
...
%left EQ NE DOT
...
%nonassoc LBRACK          /* highest precedence  */
...
BoolOp:
    GE                                  { ">="                }
  | LE                                  { "<="                }
  | GT                                  { ">"                 }
  | LT                                  { "<"                 }
  | NE                                  { "!="                }
  | EQ                                  { "=="                }
;

ExprNotAccess:
    AtExprNotAccess                     { $1                  }
  ...
  | Expr DOT BoolOp Expr DOT BoolOp Expr{ Andalso(Prim2($3, $1, $4), Prim2($6, $4, $7))}
  ...
  | Expr SEQOR  Expr                    { Orelse($1, $3)      }
;
\end{FSharp}
\caption{CPar.fsy}\label{fig:2016:opg4:cpar}
\end{figure}
\newpage
I ændringerne ovenfor kan det ses at der kun er tilføjet een ny token og det er \Verb|DOT|, dette er så vi kan understøtte i lexeren at den møder et punktum. I vores parser som kan ses i figur~\ref{fig:2016:opg4:cpar}, så er der sket lidt mere.\\\\
Der er blevet oprettet een nye grammer, \Verb|BoolOp|. \Verb|BoolOp| er til for at matche med vores understøttede boolske operationer. Der er også blevet tilføjet en ny regel til \Verb|ExprNotAccess| som sørger for at matche den nye funktionalitet, hvilket den gør ved blot at tjekke som der er 3 expressions i streg, herefter smider den dem ind i en \Verb|Andalso|.
 %Dette gør den ved at tjekke om vi blot har at gøre med et interval på 3 elementer eller flere hvis der er 2 elementer understøtter vi ikke operationen da de normale operatorer bør bruges her. Interval er en rekursiv grammer som kan lave en hægtet \Verb|Andalso| så vi kan tilknytte et vilkårligt antal elementer til vores interval check.
\subsubsection{Udvid micro-c programmet med flere eksempler}
\begin{figure}[!ht]
\begin{ccode}
void main(){
    print 2 .< 3 .< 4;
    print 3 .< 2 .== 2;
    print 3 .> 2 .== 2;
    print (3 .> 2 .== 2) == (3 .> 1 .== 1);
    print (3 .> 2 .== 2) == 1;
    // udvidelser opgave 4.2
    print 2 .< 3 .<= 4;
    print 2 .> 1 .>= 4;
    print (2 .>= 2 .>= 4) .== (2 .>= 2 .>= 4) .== (2 .>= 2 .>= 4);
    print (2 .>= 2 .>= 4) .!= (2 .<= 2 .<= 4) .== (2 .>= 2 .>= 4);
    print (2 .>= 2 .>= 4) .!= (2 .<= 2 .<= 4) .!= (2 .>= 2 .>= 4) == (2 .>= 2 .>= 4) .!= (2 .<= 2 .<= 4) .!= (2 .>= 2 .>= 4);
}
\end{ccode}
\caption{Det udvidede program}
\end{figure}
Her testes der flere ting både at alle operatorer er under og at de spiller sammen med sig selv samt eksiterende funktionalitet i koden. Resultaterne er præcis som forventet, dette kan verificeres ved at håndkøre resultaterne hvor man vil finde at svarene stemmer overens med det forventede resultat.