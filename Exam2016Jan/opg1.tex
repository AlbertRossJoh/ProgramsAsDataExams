Betragt dette regulære udtryk over alfabetet $\{k,l,v,h,s\}$:
\begin{align*}
    k(l|v|h|)+s
\end{align*}
Ved antagelse, at
\begin{align*}
    k & \text{ svarer til } & kør\\
    l & \text{ svarer til } & ligeud\\
    v & \text{ svarer til } & venstre\\
    h & \text{ svarer til } & højre\\
    s & \text{ svarer til } & slut
\end{align*}
så beskriver det regulære udtryk strenge, der symboliserer køreture, eksempelvis turen fra hjem til ITU.\@
\subsubsection{Giv nogle eksempler på strenge der genkendes af det regulære udtryk samt en uformel beskrivelse}
Nogle eksempler på strenge der genkendes af det regulære udtryk er.
\begin{itemize}
    \item kls
    \item klllllllls
    \item klvhs
    \item khhhhs
\end{itemize}
Sproget beskriver en mængde handlinger man skal tage på en køretur, man kan enten køre ligeud til venstre eller højre, der er følgende krav til en streng.
\begin{itemize}
    \item Man skal starte med $k$ og slutte med $s$
    \item Mellem $k$ og $s$ skal man \textit{mindst} een gang enten dreje eller køre ligeud.
\end{itemize}
\subsubsection{Lav udtrykket om til en NFA}
Vi kan starte med at dele udtrykket op i mindre dele og repræsentere dem via en graf visning.\\
\begin{figure}[!ht]
    \centering
    \begin{tikzpicture}[shorten >=1pt,node distance=2cm,on grid,auto] 
       \node[state, initial] (q_1)   {$1$}; 
        \node[state,accepting] (q_2) [ right =of q_1] {$2$};
              
        \path[->] 
        (q_1) edge  node {$k$} (q_2)
        ;
    \end{tikzpicture}
    \caption{$k$}\label{fig:2016:examfignfak}
\end{figure}
\newpage
\begin{figure}[!ht]
    \centering
    \begin{tikzpicture}[shorten >=1pt,node distance=2cm,on grid,auto] 
       \node[state, initial] (q_1)   {$1$}; 
        \node[state] (q_3) [ right =of q_1] {$3$};
        \node[state] (q_2) [above  =of q_3] {$2$};
        \node[state] (q_4) [below  =of q_3] {$4$};
        \node[state,accepting] (q_5) [ right =of q_3] {$5$};
              
        \path[->] 
        (q_1) edge [bend left] node {$\epsilon$} (q_2)
        (q_1) edge  node {$\epsilon$} (q_3)
        (q_1) edge [bend right] node {$\epsilon$} (q_4)
        (q_2) edge [bend left] node {$l$} (q_5)
        (q_3) edge  node {$v$} (q_5)
        (q_4) edge [bend right] node {$h$} (q_5)
        (q_5) edge [bend left=30] node {$\epsilon$} (q_1)
        ;

    \end{tikzpicture}
    \caption{$(l|v|h)+$}\label{fig:2016:examfignfalvh}
\end{figure}
\begin{figure}[!ht]
    \centering
    \begin{tikzpicture}[shorten >=1pt,node distance=2cm,on grid,auto] 
       \node[state, initial] (q_1)   {$1$}; 
        \node[state,accepting] (q_2) [ right =of q_1] {$2$};
              
        \path[->] 
        (q_1) edge  node {$s$} (q_2)
        ;
    \end{tikzpicture}
    \caption{$s$}\label{fig:2016:examfignfas}
\end{figure}
Hvis vi sætter figure~\ref{fig:2016:examfignfak}~\ref{fig:2016:examfignfalvh}~\ref{fig:2016:examfignfas} sammen så får vi følgende graf.
\begin{figure}[!ht]
    \centering
    \begin{tikzpicture}[shorten >=1pt,node distance=2cm,on grid,auto] 
        \node[state, initial] (q_1)   {$1$}; 
        \node[state] (q_2) [right=of q_1] {$2$};
        \node[state] (q_4) [ right =of q_2] {$4$};
        \node[state] (q_3) [above  =of q_4] {$3$};
        \node[state] (q_5) [below  =of q_4] {$5$};
        \node[state] (q_6) [right  =of q_4] {$6$};
        \node[state,accepting] (q_7) [ right =of q_6] {$7$};
              
        \path[->] 
        (q_1) edge node{$k$}(q_2)
        (q_2) edge [bend left] node{$\epsilon$}(q_3)
        (q_2) edge node{$\epsilon$}(q_4)
        (q_2) edge [bend right] node{$\epsilon$}(q_5)
        (q_3) edge [bend left] node{$l$}(q_6)
        (q_4) edge node{$v$}(q_6)
        (q_5) edge [bend right] node{$h$}(q_6)
        (q_6) edge node{$s$}(q_7)
        (q_6) edge [bend left=30] node{$\epsilon$}(q_2)
        ;

    \end{tikzpicture}
    \caption{The labeled NFA, representing $k(l|v|h|)+s$}\label{fig:2016:examfignfa}
\end{figure}
\newpage
\subsubsection{Lav udtrykket om til en DFA}
For at konstruerer en DFA fra NFA'en skal vi inddele automaten i subsæt og se hvilke sæt referer til andre sæt.
\begin{figure}[!ht]
    \centering
    \begin{tabular}{c|cccccc}
            &$k$&$l$&$v$&$h$&$s$&NFA state\\\hline
        $S_1$&$\{2\}^{S_2}$&\{\}&\{\}& \{\}&\{\}& $\{1\}$\\
        $S_2$&\{\}&$\{6\}^{S_3}$&$\{6\}^{S_3}$&$\{6\}^{S_3}$&\{\}&\{2,3,4,5\}\\
        $S_3$&\{\}&$\{\}^{S_3}$&$\{\}^{S_3}$&$\{\}^{S_3}$&$\{7\}^{S_4}$&\{6\}\\
        $S_4$&\{\}&$\{\}$&$\{\}$&$\{\}$&$\{\}$&\{7\}
    \end{tabular}
\end{figure}
Nu kan vi konstruerer DFA'en.
\begin{figure}[!ht]
    \centering
    \begin{tikzpicture}[shorten >=1pt,node distance=2cm,on grid,auto] 
        \node[state, initial] (q_1)   {$S_1$}; 
        \node[state] (q_2) [right=of q_1] {$S_2$};
        \node[state] (q_3) [ right =of q_2] {$S_3$};
        \node[state,accepting] (q_4) [ right =of q_3] {$S_4$};
              
        \path[->] 
            (q_1) edge node{$k$}(q_2)
            (q_2) edge  node{$lvh$}(q_3)
            (q_3) edge [loop] node{$lvh$}(q_3)
            (q_3) edge node{$s$}(q_4)
        ;
    \end{tikzpicture}
    \caption{The labeled DFA, representing $k(l|v|h|)+s$}\label{fig:2016:examfigdfa}
\end{figure}
Grunden til at automaten er deterministisk er at der ikke går to ens kanter fra een knude til to forskellige, der eksisterer heller ikke epsilon kanter i automaten angivet i figur~\ref{fig:2016:examfigdfa}.
\subsubsection{Wtf}
\begin{align*}
    (l|vl|hl)+
\end{align*}