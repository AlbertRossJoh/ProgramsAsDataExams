Betragt dette regulære udtryk over alfabetet $\{d, ','\}$, hvor $d$ står for decimalt ciffer og ',' er komma:
\begin{align}
    d+','?d*
\end{align}
Ved antagelse af, at $d$ svarer til tallene fra 0 til 9 og ',' er et komma, så beskriver det regulære udtryk kommatal.
\subsubsection{}
Udtrykket kan beskrive tal i sættet $\mathbb{R}^+_0$, dette vil altså sige vilkårligt store reelle tal.\\
Følgende eksempler er inkluderet i sættet:
\begin{itemize}
    \item $0$
    \item $1,1$
    \item $9,999999999$
    \item $20$
    \item $123456,0$
    \item $2,$
\end{itemize}
\subsubsection{}
\begin{figure}[!ht]\label{fig:examfig}
    \centering
    \begin{tikzpicture}[shorten >=1pt,node distance=2cm,on grid,auto] 
       \node[state, initial] (q_0)   {$1$}; 
        \node[state] (q_1) [ right =of q_0] {$2$};
        \node[state] (q_2) [above =of q_1] {$3$};
        \node[state] (q_3) [right =of q_1] {$4$};
        \node[state] (q_4) [right =of q_3] {$5$};
        \node[state] (q_5) [right =of q_4] {$6$};
        \node[state, accepting] (q_6) [below =of q_4] {$7$};
              
        \path[->] 
        (q_0) edge  node {$d$} (q_1)
        (q_1) edge [bend left] node {$\epsilon$} (q_2)
        (q_2) edge [bend left] node {$d$} (q_1)
        (q_1) edge node {$\epsilon$} (q_3)
        (q_3) edge [bend left] node {','} (q_4)
        (q_3) edge [bend right] node {$\epsilon$} (q_4)
        (q_4) edge node {$\epsilon$} (q_6)
        (q_4) edge [bend left] node {$\epsilon$} (q_5)
        (q_5) edge [bend left] node {$d$} (q_4)
        ;
    \end{tikzpicture}
    \caption{The labeled NFA representing $d+$}
\end{figure}
\subsubsection{Vil tilstandsmaskinen acceptere netop de strenge, som genkendes af det regulære udtryk ovenfor}
Ja man kan dele det regulære udtryk op i diskrete dele. Dette kan vi sammenligne med tilstandsmaskinen angivet nedenfor.\\
\begin{figure}[!ht]\label{fig:dplus}
    \centering
    \begin{tikzpicture}[shorten >=1pt,node distance=2cm,on grid,auto] 
       \node[state, initial] (q_0)   {$1$}; 
        \node[state, accepting] (q_1) [ right =of q_0] {$2$};
        \node[state] (q_2) [above =of q_1] {$3$};
              
        \path[->] 
        (q_0) edge  node {$d$} (q_1)
        (q_1) edge [bend left] node {$\epsilon$} (q_2)
        (q_2) edge [bend left] node {$d$} (q_1)
        ;
    \end{tikzpicture}
    \caption{The labeled NFA representing $d+$}
\end{figure}
\begin{figure}[!ht]\label{fig:comma}
    \centering
    \begin{tikzpicture}[shorten >=1pt,node distance=2cm,on grid,auto] 
       \node[state, initial] (q_0)   {$1$}; 
        \node[state, accepting] (q_1) [ right =of q_0] {$2$};
              
        \path[->] 
        (q_0) edge [bend right] node {','} (q_1)
        (q_0) edge [bend left] node {$\epsilon$} (q_1)
        ;
    \end{tikzpicture}
    \caption{The labeled NFA representing $','?$}
\end{figure}
\begin{figure}[!ht]\label{fig:dopt}
    \centering
    \begin{tikzpicture}[shorten >=1pt,node distance=2cm,on grid,auto] 
       \node[state, initial] (q_0)   {$1$}; 
        \node[state, accepting] (q_1) [ right =of q_0] {$2$};
              
        \path[->] 
        (q_1) edge [bend left] node {$d$} (q_0)
        (q_0) edge [bend left] node {$\epsilon$} (q_1)
        ;
    \end{tikzpicture}
    \caption{The labeled NFA representing $d*$}
\end{figure}
Det er åbenlyst at se at disse alle er dele af den tilstandsmaskine angivet.\\\\
Tilstandsmaskinen angivet er ikke deterministisk da man kan være i flere knuder på en gang på grund af eksistensen af epsilon kanter.
\newpage
\subsubsection{}
\begin{align}
    (d+','?d*)?
\end{align}
\subsubsection{}
Løsningen kan findes i kommatal.fsl